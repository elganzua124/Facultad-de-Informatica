\documentclass[a4paper,10pt]{article}


\RequirePackage{color,graphicx}
\usepackage[usenames,dvipsnames]{xcolor}
\usepackage[big]{layaureo} 				%better formatting of the A4 page
% an alternative to Layaureo can be ** \usepackage{fullpage} **
\usepackage{supertabular} 				%for Grades
\usepackage{titlesec}					%custom \section
%Setup hyperref package, and colours for links
\usepackage{hyperref}
\definecolor{linkcolour}{rgb}{0,0.2,0.6}
\hypersetup{colorlinks,breaklinks,urlcolor=linkcolour, linkcolor=linkcolour}
\usepackage[utf8]{inputenc}

%Sections inspired by:
%http://stefano.italians.nl/archives/26
\titleformat{\section}{\Large\scshape\raggedright}{}{0em}{}[\titlerule]
\titlespacing{\section}{0pt}{3pt}{3pt}
%Tweak a bit the top margin
%\addtolength{\voffset}{-1.3cm}

%Italian hyphenation for the word: ''corporations''
\hyphenation{im-pre-se}

%-------------WATERMARK TEST [**not part of a CV**]---------------
\usepackage[absolute]{textpos}

\setlength{\TPHorizModule}{30mm}
\setlength{\TPVertModule}{\TPHorizModule}
\textblockorigin{2mm}{0.65\paperheight}
\setlength{\parindent}{0pt}

\usepackage{vmargin}

\setpapersize{A4}
\setmargins{3.5cm}       % margen izquierdo
{2.0cm}                        % margen superior
{14.5cm}                      % anchura del texto
{24.0cm}                    % altura del texto
{10pt}                           % altura de los encabezados
{1cm}                           % espacio entre el texto y los encabezados
{0pt}                             % altura del pie de página
{2cm}                           % espacio entre el texto y el pie de página


%%%%%%%%%%%%%%%%%%%%%%%%%%%%%%%%%%%%%%%%%%%%%%%%%%%%%%%%%%%%%%%%%%

\title{Cuestionario para los usuarios del Blog unagauchada}
\author{www.unagauchada.tumblr.com}
\date{}

\begin{document}

\maketitle

\renewcommand{\labelitemi}{$\bigcirc$}

\section{Cuestionario}
% En esta sección se escribe cada una de las respuestas 
% luego de cada una de las preguntas con un previo salto de linea
% \\
$\hookrightarrow{}$ Para contestar cada pregunta, indique su respuesta rellenando el circulo correspondiente a la opción que desea elegir. Marque solo una opción.
\begin{enumerate}
    \item ¿Qué tan conforme estás con el blog?
    \begin{itemize}
        \item Muy conforme
        \item Conforme
        \item Satisfecho
        \item Poco satisfecho
        \item Inconforme
        \item Muy inconforme
    \end{itemize}
    
    \item ¿Te gustaría tener un perfil para mostrarle a la gente más datos personales?
    \begin{itemize}
        \item Si
        \item No
    \end{itemize}
    
    \item ¿Te gustaría conocer más información sobre los gauchos? (Experiencia, favores realizados, $\ldots$)
    \begin{itemize}
        \item Si
        \begin{itemize}
            \item ¿Qué información te gustaría conocer de los gauchos? 
            
            \rule{100mm}{0.1mm} \\
            \rule{100mm}{0.1mm} \\
            \rule{100mm}{0.1mm}
        \end{itemize}
        \item No
    \end{itemize}
    
    \item ¿Te gustaría valorar a los gauchos que te hayan realizado algún favor?
    \begin{itemize}
        \item Si
        \item No
    \end{itemize}
    
    \item ¿Cuántos favores has recibido?
    \begin{itemize}
        \item Muchos (más de 10)
        \item Bastante (más de 6)
        \item Algunos (más de 3)
        \item Pocos (1 ó 2)
        \item Ninguno
    \end{itemize}
    
    \item ¿Cuántos favores realizaste?
    \begin{itemize}
        \item Muchos (más de 10)
        \item Bastante (más de 6)
        \item Algunos (más de 3)
        \item Pocos (1 ó 2)
        \item Ninguno
    \end{itemize}
    
    \item ¿Con cuanta frecuencia visitás el blog?
    \begin{itemize}
        \item Todo los días
        \item Día por medio
        \item Una vez a la semana
        \item Una vez al mes
    \end{itemize}
    
    \item ¿Qué edad tenés?
    \begin{itemize}
        \item Mayor de 65
        \item Entre 31 y 65
        \item Entre 18 y 30
        \item Menor de 18
    \end{itemize}
    
    \item ¿Recomendarías el blog?
    \begin{itemize}
        \item Si
        \item No
        \begin{itemize}
            \item ¿Por qué no lo recomendarías?
            
            \rule{100mm}{0.1mm} \\
            \rule{100mm}{0.1mm} \\
            \rule{100mm}{0.1mm}
        \end{itemize}
    \end{itemize}
    
\end{enumerate}

\end{document}
